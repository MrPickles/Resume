\documentclass[]{template}

%%%%%%%%%%%%%%%%%%%%%%%%%%%%%%%%%%%%%%%%%%%%%%%%%%%%%%%%%%%%%%%%%%%%%%%%%%%%%%%%
% "Style Guide"
%
% - 80 character line limit
% - For bulleted paragraphs, put the text in a newline after the \item and keep
%   a consistent two-space indentation.
% - For line wraps anywhere else, make the indentation align with where the text
%   started after the initial tag.
% - Put newlines (double backslash) on their own lines for readability.
%%%%%%%%%%%%%%%%%%%%%%%%%%%%%%%%%%%%%%%%%%%%%%%%%%%%%%%%%%%%%%%%%%%%%%%%%%%%%%%%

\usepackage{graphicx,xcolor,fontawesome}

\begin{document}

\namesection{Andrew Liu}
{
  \faHome \, \urlstyle{same}\url{https://andrew.cloud/}
  \textcolor{white}{\textbullet} % invisible bullet for spacing
  \faGithub \, \href{https://github.com/MrPickles}{MrPickles}
  \\ % newline for spacing
  \textcolor{white}{\textbullet} % invisible bullet for spacing
  \faEnvelope \, \href{mailto:aliu1@umd.edu}{aliu1@umd.edu}
  \textcolor{white}{\textbullet} % invisible bullet for spacing
  \faLinkedin \, \href{https://linkedin.com/in/liuandrewk}{liuandrewk}
  \textcolor{white}{\textbullet} % invisible bullet for spacing
  \faPhone \, \href{tel:2029059225}{202-905-9225}
}

\section{Education}

\runsubsection{University of Maryland, College Park}
\hfill
\location{Sept 2013 - May 2017 | College Park, MD}
\begin{tightemize}
\item BS in Computer Science, minor in Statistics
\item GPA: 3.97/4.0 --- Magna Cum Laude
\item ACES Cybersecurity Honors Program graduate
      and President's Scholarship recipient
\end{tightemize}

\section{Experience}

\runsubsection{Google}
\descript{Software Engineer}
\hfill
\location{August 2017 – Present | Mountain View, CA}
\begin{tightemize}
\item
  Built risk-based re-authentication integrations into Google servers, scanning
  over 400 queries-per-second of potential hijacker traffic via RPC and
  opportunistically serving suspicious users second factor auth challenges.
\item
  Maintained internal microservices that exposed authentication APIs over HTTP
  and RPC. Consulted and collaborated with teams looking to integrate with these
  APIs, leading to public launches such as LDAP-as-a-service for G Suite
  customers and two-factor authentication in Google Cloud virtual machines.
\item
  Added UI support on the login page for sending Google Prompts to the iOS Gmail
  app. Tripled Google Prompt coverage for iOS users to 140 million users.
\item
  Designed and prototyped a end-to-end login challenge where users could scan QR
  codes to approve their sign-in attempts from a remote browser. Created
  challenge pages to generate QR codes and approve logins using Soy and
  JavaScript, and maintained approval transaction state across several Java
  servers.
\item
  Conducted A/B tests and analyzed data on various authentication features.
  Wrote Jupyter Notebooks to communicate experiment results with teammates.
  Improved pass rate of Google Prompts by 10\% and overall success rate for
  Google sign-in by 5\%, through the aforementioned experiments.
\item
  Migrated Google Authenticator's build system to Bazel and removed all of the
  app's internal dependencies. Currently in the process of releasing an open
  source version of the Android app.
\item
  \techstack{
    Java,
    C++,
    Python,
    JavaScript,
    SQL,
    Soy,
    Guice,
    Dagger,
    Bazel,
    gRPC,
    Jupyter,
    Git,
    Mercurial
  }
\end{tightemize}

\sectionsep

\runsubsection{Dropbox}
\descript{Software Engineering Intern}
\hfill
\location{May 2016 – Aug 2016 | San Francisco, CA}
\begin{tightemize}
\item
  Investigated and analyzed performance issues with Dropbox's Python web server.
  Fixed server inefficiencies, reducing the loading time of Dropbox's
  website homepage by 33\% as a result of the optimizations.
\item
  Implemented an end-to-end flow written with React.js and Python to let users
  request team upgrades to Dropbox Business. Integrated with an RPC service to
  send email campaigns to Dropbox team admins.
\item
  \techstack{
    Python,
    React.js,
    Pxyl,
    Vagrant,
    Git
  }
\end{tightemize}

\sectionsep

\runsubsection{Google}
\descript{Software Engineering Intern}
\hfill
\location{May 2015 - Aug 2015 | Kirkland, WA}
\begin{tightemize}
\item
  Created an exploitability detection engine in Breakpad (an open source crash
  reporting suite written in C++) that predicted whether a binary crash occurred
  due to attacker exploitation by implementing triaging heuristics such as
  buffer overflow detection and binary permission checking.
\item
  Open sourced all internal changes to Breakpad for public usage and auditing.
\end{tightemize}

\sectionsep

\runsubsection{US Army Research Laboratory}
\descript{Security Research Intern}
\hfill
\location{May 2014 - Aug 2014 | Adelphi, MD}
\begin{tightemize}
\item
  Coauthored a research paper for the USENIX security conference in 2015,
  detailing the usage of code stylometry and machine learning as a means of
  authorship attribution for anonymous samples of source code.
\item
  Wrote a Python web crawler to scrape historical source code entries from
  programmers participating in the Google Code Jam programming competition.
  Cleaned up scraped source code for use as a corpus of training data.
\end{tightemize}

\section{Skills}

\subsection{Languages}
Java,
Kotlin,
Python,
SQL,
HTML/CSS,
JavaScript,
TypeScript,
Soy,
C/C++,
Ruby,
Go,
\LaTeX,
R

\sectionsep

\subsection{Technologies}
Git,
Mercurial,
gRPC,
Guice,
Dagger,
Jupyter/Colaboratory,
React.js,
Vue.js,
PostgreSQL,
MongoDB,
Wireshark,
IDA Pro,
GDB

\lastupdated

\end{document}
\documentclass[]{article}
